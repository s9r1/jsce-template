\documentclass{jjsce}
%\documentclass[jscefinal]{jjsce}% 登載可決定後の最終原稿を提出するときはこちら → ページ番号なしになる

\usepackage{amsmath}
\usepackage{amsthm}
\usepackage[defaultsups]{newtxtext}
\usepackage[varg]{newtxmath}
\usepackage{bm}% 数式でボールドイタリックを使いたいとき
\usepackage[dvipdfmx]{graphicx}
%\usepackage{xcolor}
\usepackage[superscript]{cite}
\usepackage{url}
\usepackage{endnotes}
\usepackage[savepos]{zref}
% \usepackage[dvipdfmx]{hyperref}
% \usepackage{pxjahyper}
\aboveEtitlesep20mm% 最終ページの英文タイトル部:本文末の並行止めした後のスペース調整.詳細はreadme.pdf参照.

\usepackage{jjsce-macros}

\usepackage{subfiles}

\graphicspath{{fig}}

\begin{document}
\jtitle{土木学会論文集和文原稿作成例}
%\jsubtitle{}
\etitle{FORMATTING JAPANESE MANUSCRIPT FOR JOURNALS OF JSCE}
%\esubtitle{}
\authorlist{%
  \authorentry{土木 太郎}{Taro DOBOKU}{a}
  \authorentry{四谷 花子}{Hanako YOTSUYA}{b}
}
\Caffiliate[a]{正会員 土木大学教授 工学部土木工学科
  (\jipcode{160--0004}東京都新宿区四谷一丁目無番地)}{doboku@jsce.ac.jp}
\affiliate[b]{正会員 土木建設株式会社 技術開発部
  (\jipcode{160--0004}東京都新宿区三矢六丁目13-5)}{hanako@jsce.co.jp }

%\Jbreakauthorline{4}
%\breakauthorline{4}
% \received{2022}{1}{31}% Received Date を記入
% \accepted{2022}{4}{28}% Accepted Date を記入
\begin{abstract}
  要旨の長さは350字以内です.キーワードは5つ程度書いて下さい.
\end{abstract}
\begin{Eabstract}
  The length of English abstract should be 300 words or less.
\end{Eabstract}
\begin{keyword}
  times, italic, 10pt, one blank line below abstract, indent if key words exceed one line
\end{keyword}

\maketitle

\subfile{sec1}
% \subfile{sec2}

\Acknowledgment
「謝辞」は「結論」の後に置いて下さい.

\bibliographystyle{jsce}
\bibliography{main}

\end{document}
